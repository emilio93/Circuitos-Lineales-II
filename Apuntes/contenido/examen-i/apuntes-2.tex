%   Copyright 2016 Emilio Rojas
%
%   Licensed under the Apache License, Version 2.0 (the "License");
%   you may not use this file except in compliance with the License.
%   You may obtain a copy of the License at
%
%       http://www.apache.org/licenses/LICENSE-2.0
%
%   Unless required by applicable law or agreed to in writing, software
%   distributed under the License is distributed on an "AS IS" BASIS,
%   WITHOUT WARRANTIES OR CONDITIONS OF ANY KIND, either express or implied.
%   See the License for the specific language governing permissions and
%   limitations under the License.

\subsection{Circuito RLC en serie}
\begin{figure}[H]
  \begin{center}
    \begin{circuitikz}
      \draw (0,0)
      to [sV, l=$v_i(t)$] (0,3)
      to[R, l_=$R$, v^<=$v_R(t)$, i_=$i_R(t)$] (3,3)
      to[L, l_=$L$, v^<=$v_L(t)$, i_=$i_L(t)$] (6,3)
      to [C, l_=$C$, v^<=$v_C(t)$, i_=$i_C(t)$] (9,3)
      to [short, i^=$i_i(t)$] (9,0)
      to [short] (0,0);
    \end{circuitikz}
  \end{center}
\end{figure}

\begin{equation*}
  v_i(t) = V_0 \cos(\omega t) = v_R(t) + v_L(T) + v_C(t)
\end{equation*}

Sabemos que la corriente va a sufrir un desfase debido a que la magnitud del
desfase producido por el capacitor se verá contrarrestado por el del inductor.
También sabemos que se verá afectada su magnitud. Por esto la expresión para la
corriente queda en su forma más generalizada, en que la frecuencia no varia pero
su magnitud y fase pueden variar:
\begin{align*}
  i_i(t) &= I_M \cos(\omega t + \phi) = i_R(t) = i_L(t) = i_C(t) \\
  v_i(t-\phi) &= i_i(t-\phi) R + L \frac{di_i(t-\phi)}{dt} + \frac{1}{C} \int i_i(t-\phi) dt \\
  v_i(t-\phi) &= I_0 R \cos(\omega t)  - \omega L I_0 \sin(\omega t) + \frac{I_0}{\omega C} \sin(wt) \\
  v_i(t-\phi) &= I_0 \left[R \cos(\omega t) + \left(\frac{1}{\omega C} - \omega L \right) \sin(wt) \right]
\end{align*}

Ahora creamos el siguente triangulo para reducir la expresión a algo que nos
sirva más:

\begin{center}
  \begin{tikzpicture}
    \draw
    (0,0) coordinate (a)
    -- (1.5,0) coordinate (label_adjacent)
    node[below] {$\frac{V_0}{\omega L} - \omega C V_0$}
    -- (3,0) coordinate (b)
    -- (1.5, 1.5) coordinate (label_hypotenuse)
    node[above right] {$\sqrt{R^2 + \left(\frac{1}{\omega C} - \omega L\right)^2}$}
    -- (0,3) coordinate (c)
    -- (0,1.5) coordinate (label_opposite)
    node[left] {$\frac{V_0}{R}$}
    -- (0,0) coordinate (d) node[above right] {}
    pic["$\beta$", draw=black, angle eccentricity=1.2, angle radius=1cm]
    {angle=c--b--a};
  \end{tikzpicture}
\end{center}

Notese que el valor de la hipotenusa depende únicamente de los catetos.

\begin{align*}
  \cos(\beta) &= \frac{\frac{1}{\omega C} - \omega L}{\sqrt{R^2 + \left(\frac{1}{\omega C} - \omega L\right)^2}} \\
  \sin(\beta) &= \frac{R}{\sqrt{R^2 + \left(\frac{1}{\omega C} - \omega L\right)^2}} \\
  \beta &= \tan{^{-1}}\left(\frac{R}{\frac{1}{\omega C} - \omega L}\right)
\end{align*}

Entonces ahora sustituimos:

\begin{align*}
  v_i(t-\phi) &= I_0 \left[\sqrt{R^2 + \left(\frac{1}{\omega C} - \omega L\right)^2} \sin(\beta) \cos(\omega t) + \sqrt{R^2 + \left(\frac{1}{\omega C} - \omega L\right)^2} \cos(\beta) \sin(wt) \right] \\
  v_i(t-\phi) &= I_0 \sqrt{R^2 + \left(\frac{1}{\omega C} - \omega L\right)^2} \left[ \sin(\beta) \cos(\omega t) + \cos(\beta) \sin(wt) \right]\\
  &\mathrm{Utilizando\quad}\ref{id_trig1}\\
  v_i(t-\phi) &= I_0 \sqrt{R^2 + \left(\frac{1}{\omega C} - \omega L\right)^2} \sin(\omega t - \phi) \\
  v_i(t-\phi) &= I_0 \sqrt{R^2 + \left(\frac{1}{\omega C} - \omega L\right)^2} \cos(\omega t + \beta - 90\degree),\quad \phi = 90\degree - \beta
\end{align*}

Ahora tenemos la siguiente relaciones:

\begin{align*}
  I_M &= I_0 \sqrt{R^2 + \left(\frac{1}{\omega C} - \omega L\right)^2} = \frac{V_0}{\sqrt{R^2 + \left(\frac{1}{\omega C} - \omega L\right)^2}}\\
  i_i(t) &= \frac{V_0}{\sqrt{R^2 + \left(\frac{1}{\omega C} - \omega L\right)^2}} \cos\left(\omega t + 90\degree - \tan{^{-1}}\left(\frac{R}{\frac{1}{\omega C} - \omega L}\right)\right)
\end{align*}

\subsection*{$P_{Prom}$  en R}

\begin{align*}
  P_{Prom} &= i_i^2(t) \cdot R = \frac{V_0^2 R}{R^2 + \left(\frac{1}{\omega C} - \omega L\right)^2} \cos{^2}\left(\omega t + 90\degree - \tan{^{-1}}\left(\frac{R}{\frac{1}{\omega C} - \omega L}\right)\right) \\
  P_{Prom} &= \frac{V_0^2 R}{R^2 + \left(\frac{1}{\omega C} - \omega L\right)^2} + \left[\frac{1}{2} + \frac{1}{2}\cos\left(\omega t + 90\degree - \tan{^{-1}}\left(\frac{R}{\frac{1}{\omega C} - \omega L}\right)\right)\right] \\
  P_{Prom} &= \frac{V_0^2 R}{2\left( R^2 + \left(\frac{1}{\omega C} - \omega L\right)^2\right)}
\end{align*}

\subsection*{$P_{Prom}$ en Fuente}

\begin{align*}
  P_{Prom} &= \frac{V_0 I_M}{2} \cos(\theta)\\
  P_{Prom} &= \frac{V_0^2 \cos(\theta)}{2\left( R^2 + \left(\frac{1}{\omega C} - \omega L\right)^2\right)}\\
  P_{Prom} &= \frac{V_0^2 R}{2\sqrt{R^2 + \left(\frac{1}{\omega C} - \omega L\right)^2}}
\end{align*}


\chapter{Forma Cartesiana, Polar, Exponencial y Fasorial}


\section*{Forma Cartesiana(Rectangular)}
\begin{equation*}
  C = a + j b
\end{equation*}
Donde $C$ es un número complejo, $a$ es la parte real y $b$ la parte imaginaria,
notese que $j$ corresponde a $\sqrt{-1}$.

\section*{Forma Polar}
\begin{equation*}
  C = r\left[cos(\theta) + j\sen(\theta)\right]
\end{equation*}
\begin{equation*}
  r = \sqrt{a^2 + b^2}
\end{equation*}
\begin{equation*}
  \theta = \tan{^1}\left(\frac{b}{a}\right)
\end{equation*}

\section*{Forma Exponencial}
\begin{equation*}
  C = r e^{j\theta},\quad e^{j\theta} = cos(\theta) + j\sen(\theta)
\end{equation*}

\section*{Forma Fasorial}
\begin{equation*}
  C = r \angle \theta
\end{equation*}

\textbf{Nota}: El fasor es u número compuesto que representa la magnitud y la
fase de un senoide. El termino fasor se utiliza en vez de vector porque el
ángulo se basa en tiempo más espacio(se utilizan o analizan fuentes con misma
frecuencia).

\section{Algunas propiedades}
Suma
\begin{equation*}
  z_1 + z_2 = (x_1 + x_2) + j(y_1 + y_2)
\end{equation*}

Resta
\begin{equation*}
  z_1 - z_2 = (x_1 - x_2) + j(y_1 - y_2)
\end{equation*}

Multiplicación
\begin{equation*}
  z_1 \times z_2 = (r_1 \times r_2) \angle (\phi_1 + \phi_2)
\end{equation*}

División
\begin{equation*}
  z_1 \times z_2 = \frac{r_1}{r_2} \angle (\phi_1 - \phi_2)
\end{equation*}

Inverso
\begin{equation*}
  \frac{1}{z} = \frac{1}{r} \angle (\phi)
\end{equation*}

Raiz Cuadrada
\begin{equation*}
  \sqrt{z} = \sqrt{r} \angle \frac{\phi}{2}
\end{equation*}

Conjugado Complejo
\begin{equation*}
  z^* = x-jy = r\angle -\phi = r e^{-j\phi}
\end{equation*}

Se cumple además que
\begin{equation*}
  C \cdot C^* = r^2
\end{equation*}

\section{Relaciones Fasoriales de elementos de circuitos}

\subsection*{Resistencia}

\begin{figure}[H]
  \begin{center}
    \begin{circuitikz}
      \draw (0,0)
      to [open, v^<=$v_i(t)$, american voltages] (0,3)
      to [short, *-, i=$i_R(t)$] (3,3)
      to [R, l_=R, v^<=$v_R(t)$] (3,0)
      to [short, -*] (0, 0);

      \draw (6,0)
      to [open, v^<=$\overline{V}$, american voltages] (6, 3)
      to [short, *-, i=$\overline{I}$] (9,3)
      to [R, l_=R, v^<=$\overline{V}_R$] (9,0)
      to [short, -*] (6, 0);
    \end{circuitikz}
  \end{center}
  \caption{Diagrama de resistencia en dominio del tiempo y dominio de la frecuencia}
\end{figure}

\begin{tikzpicture}
  \begin{axis}[
    axis x line=center,
    axis y line=center,
    xlabel={$\mathrm{Re}$},
    ylabel={$\mathrm{Im}$},
    xtick={0, 1},
    xticklabels={$$, $$},
    ytick={-1, 1},
    yticklabels={$$,$$},
    xlabel style={below right},
    ylabel style={above left},
    xmin=-0.2,
    xmax=1.2,
    ymin=-0.2,
    ymax=1.2,
    smooth]
    \addplot[->, domain=0:1, mark=none, color=blue]
    {x} node[above, pos=0.5] {$\overline{V}$};
    \addplot[->, domain=0:0.5, mark=none, color=black, thick]
    {x} node[below, pos=0.5] {$\overline{I}$};
  \end{axis}
  \label{img_fasor_r}
\end{tikzpicture}


\subsection*{Inductor}

\begin{figure}[H]
  \begin{center}
    \begin{circuitikz}
      \draw (0,0)
      to [open, v^<=$v_i(t)$, american voltages] (0,3)
      to [short, *-, i=$i_L(t)$] (3,3)
      to [L, l_=$L$, v^<=$v_L(t)$] (3,0)
      to [short, -*] (0, 0);

      \draw (6,0)
      to [open, v^<=$\overline{V}$, american voltages] (6, 3)
      to [short, *-, i=$\overline{I}$] (9,3)
      to [L, l_=$L$, v^<=$\overline{V}_L$] (9,0)
      to [short, -*] (6, 0);
    \end{circuitikz}
  \end{center}
  \caption{Diagrama de inductor en dominio del tiempo y dominio de la frecuencia}
\end{figure}

\begin{tikzpicture}
  \begin{axis}[
    axis x line=center,
    axis y line=center,
    xlabel={$\mathrm{Re}$},
    ylabel={$\mathrm{Im}$},
    xtick={0, 1},
    xticklabels={$$, $$},
    ytick={-1, 1},
    yticklabels={$$,$$},
    xlabel style={below right},
    ylabel style={above left},
    xmin=-1.2,
    xmax=1.2,
    ymin=-0.2,
    ymax=1.2,
    smooth]
    \addplot[<-, domain=-1:0, mark=none, color=blue]
    {-x} node[above, pos=0.5] {$\overline{V}$};
    \addplot[->, domain=0:0.5, mark=none, color=black, thick]
    {x} node[below, pos=0.5] {$\overline{I}$};
  \end{axis}
  \label{img_fasor_r}
\end{tikzpicture}

\subsection*{Capacitor}
\begin{figure}[H]
  \begin{center}
    \begin{circuitikz}
      \draw (0,0)
      to [open, v^<=$v_i(t)$, american voltages] (0,3)
      to [short, *-, i=$i_C(t)$] (3,3)
      to [C, l_=C, v^<=$v_C(t)$] (3,0)
      to [short, -*] (0, 0);

      \draw (6,0)
      to [open, v^<=$\overline{V}$, american voltages] (6, 3)
      to [short, *-, i=$\overline{I}$] (9,3)
      to [C, l_=C, v^<=$\overline{V}_C$] (9,0)
      to [short, -*] (6, 0);
    \end{circuitikz}
  \end{center}
  \caption{Diagrama de capacitor en dominio del tiempo y dominio de la frecuencia}
\end{figure}

\begin{tikzpicture}
  \begin{axis}[
    axis x line=center,
    axis y line=center,
    axis line style={->},
    xlabel={$\mathrm{Re}$},
    ylabel={$\mathrm{Im}$},
    xtick={0, 1},
    xticklabels={$$, $$},
    ytick={-1, 1},
    yticklabels={$$,$$},
    xlabel style={below right},
    ylabel style={above left},
    xmin=-1.2,
    xmax=1.2,
    ymin=-0.2,
    ymax=1.2,
    smooth]

    \addplot[->, domain=0:1, mark=none, color=blue]
    {x} node[above, pos=0.5] {$\overline{V}$};
    \addplot[<-, domain=-0.5:0, mark=none, color=black, thick]
    {-x} node[below, pos=0.5] {$\overline{I}$};
  \end{axis}
  \label{img_fasor_r}
\end{tikzpicture}

\subsection*{Resumen}

\begin{tabular}{c | c | c}
  Elemento & Dominio del tiempo & Dominio de la frecuencia\\ \hline
  & & \\
  R & $v(t)=Ri(t)$ & $\overline{V} = \overline{I}R$\\
  & & \\
  L & $v(t)=L\frac{di(t)}{dt}$ & $\overline{V} = j\omega L\overline{I}$\\
  & & \\
  C & $i(t)=C\frac{dv(t)}{dt}$ & $\overline{V} = \frac{\overline{I}}{j\omega C}$
\end{tabular}
