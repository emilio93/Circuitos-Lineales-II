\section{Desfase, Adelanto, Atraso}

\begin{tikzpicture}
  \begin{axis}[
    axis x line=center,
    axis y line=center,
    xlabel={$t$},
    ylabel={$f(t)$},
    xtick={
    0,
    1.5707963,
    3.1415926,
    4.7123889,
    6.2831853},
    xticklabels={
    0,
    $\frac{\pi}{2}$,
    $\pi$,
    $3\frac{\pi}{2}$,
    $2\pi$},
    ytick={-1, 1},
    yticklabels={$-V_0$,$V_0$},
    xlabel style={below right},
    ylabel style={above left},
    xmin=-0.2,
    xmax=7,
    ymin=-1.4,
    ymax=1.4,
    xscale=2,
    smooth]
    \addplot[domain=0:2*pi, mark=none, color=blue]
    {sin(deg(x))} node[above, pos=0.25] {$V_1 = V_0 \sin(t)$};
    \addplot[domain=0:2*pi, mark=none, color=red]
    {sin(deg(x+pi/2))} node[below, pos=0.5] {$V_2 = V_0 \sin(t+\phi)$};
  \end{axis}
  \label{img_desfase}
\end{tikzpicture}

Las señales están desfasadas.

$V_2$ adelanta a $V_1$ en $\phi$.
$V_1$ se atrasa de $V_2$ en $\phi$.

Si $\phi=0\degree$ no hay desfase.

\section{Identidades Trigonométricas}
\begin{align}
  \sin(A \pm B) &= \sin(A) \cdot \cos(B) \pm \cos(A) \cdot \sin(B) \label{id_trig1} \\
  \cos(A \pm B) &= \cos(A) \cdot \cos(B) \mp \sin(A) \cdot \sin(B) \label{id_trig2}
\end{align}

\begin{align}
  \sin(A \pm 180\degree) &= -\sin(A) \label{id_trig3} \\
  \cos(A \pm 180\degree) &= -\cos(A) \label{id_trig4} \\
  \sin(A \pm 90\degree) &= \pm \sin(A) \label{id_trig5} \\
  \cos(A \pm 90\degree) &= \mp \cos(A) \label{id_trig6}
\end{align}

\begin{align}
  \cos^{2}(A) = \frac{1 + \cos(2 A)}{2} \label{id_trig7}
\end{align}

\section{Elementos Pasivos}

\subsection*{Resistencia}

\begin{figure}[ht]
  \begin{center}
    \begin{circuitikz}
      \draw (0,0)
      to [short, *-] (0.5, 0)
      to [R, l^=R, v_=$v_R(t)$, i^=$i_R(t)$] (3.5, 0)
      to [short, -*] (4, 0);
    \end{circuitikz}
  \end{center}
\end{figure}

\begin{align}
  v_R(t) &= R \cdot i_R(t) \label{v_r} \\
  i_R(t) &= \frac{v_R(t)}{R} \label{i_r}
\end{align}

\subsection*{Inductor}

\begin{figure}[ht]
  \begin{center}
    \begin{circuitikz}
      \draw (0,0)
      to [short, *-] (0.5, 0)
      to [L, l^=L, v_=$v_L(t)$, i^=$i_L(t)$] (3.5, 0)
      to [short, -*] (4, 0);
    \end{circuitikz}
  \end{center}
\end{figure}

\begin{align}
  v_L(t) &= L \cdot \frac{di_L(t)}{dt}\label{v_l} \\
  i_L(t) &= \frac{1}{L} \int v_L(t) dt \label{i_l}
\end{align}

\subsection*{Capacitor}

\begin{figure}[ht]
  \begin{center}
    \begin{circuitikz}
      \draw (0,0)
      to [short, *-] (0.5, 0)
      to [C, l^=C, v_=$v_C(t)$, i^=$i_C(t)$] (3.5, 0)
      to [short, -*] (4, 0);
    \end{circuitikz}
  \end{center}
\end{figure}

\begin{align}
  v_C(t) &= \frac{1}{C} \int i_C(t) dt \label{v_c} \\
  i_C(t) &= C \cdot \frac{dv_C(t)}{dt} \label{i_c}
\end{align}

\section{Potencia Promedio}

\begin{align}
  P(t) = i(t) \cdot v(t) = \frac{v^2(t)}{R} = \frac{i^2(t)}{R} \label{p_prom}
\end{align}

\subsection{Resistencia}

\begin{figure}[ht]
  \begin{center}
    \begin{circuitikz}
      \draw (0,0)
      to [sV, l=$v_i(t)$] (0,3)
      to [short, i^=$i_R(t)$] (2,3)
      to [R, l_=$R$, v^=$v_R(t)$] (2, 0)
      to [short] (0,0);
    \end{circuitikz}
  \end{center}
\end{figure}


\begin{align*}
  v_i(t) &= V_0 \cos(\omega t)
\end{align*}

Utilizando \ref{i_r}:
\begin{align*}
  i_R(t) &= \frac{V_0 \cos(\omega t)}{R}
\end{align*}

Utilizando \ref{p_prom}:
\begin{align*}
  P_R(t) &= \frac{V_0^2 \cos^{2}(\omega t)}{R}
\end{align*}

Utilizando \ref{id_trig7}:
\begin{align*}
  P_R(t) &= \frac{V_0^2 + V_0^2 \cos(2 \omega t)}{2 R} \\
  P_{R_{Prom}}(t) &= \frac{V_0^2}{2 R}
\end{align*}

\subsection{Inductor}

\begin{figure}[ht]
  \begin{center}
    \begin{circuitikz}
      \draw (0,0)
      to [sV, l=$v_i(t)$] (0,3)
      to [short, i^=$i_L(t)$] (2,3)
      to [L, l_=$L$, v^=$v_L(t)$] (2, 0)
      to [short] (0,0);
    \end{circuitikz}
  \end{center}
\end{figure}

\begin{align*}
  v_i(t) &= V_0 \cos{\omega t}
\end{align*}

Utilizando \ref{i_l}:
\begin{align*}
  i_L(t) &= \frac{V_0 \sin(\omega t)}{\omega L} \\
  i_L(t) &= \frac{V_0 \cos(\omega t - 90 \degree)}{\omega L}
\end{align*}

Utilizando \ref{p_prom}:
\begin{align*}
  P_L(t) &= \frac{V_0^2 \sin(\omega t) \cos(\omega t)}{\omega L} \\
  P_L(t) &= \frac{V_0^2 \sin(2 \omega t)}{2 \omega L} \\
  P_{L_{Prom}}(t) &= 0
\end{align*}

\textbf{Nota:} El inductor adelanta la corriente $90 \degree$.

\subsection{Capacitor}
\begin{figure}[ht]
  \begin{center}
    \begin{circuitikz}
      \draw (0,0)
      to [sV, l=$v_i(t)$] (0,3)
      to [short, i^=$i_C(t)$] (2,3)
      to [C, l_=$C$, v^=$v_C(t)$] (2, 0)
      to [short] (0,0);
    \end{circuitikz}
  \end{center}
\end{figure}

\begin{align*}
  v_i(t) &= V_0 \cos(\omega t)
\end{align*}

Utilizando \ref{i_c}:
\begin{align*}
  i_C(t) &= - \omega C V_0 \sin(\omega t) \\
  i_C(t) &= \omega C V_0 \cos(\omega t + 90 \degree)
\end{align*}

Utilizando \ref{p_prom}:
\begin{align*}
  P_C(t) &= - \omega C V_0^2 \sin(\omega t) \cos(\omega t) \\
  P_C(t) &= - \frac{\omega C V_0^2 \sin(2 \omega t)}{2} \\
  P_{C_{Prom}}(t) &= 0
\end{align*}

\textbf{Nota:} El capacitor atrasa la corriente $90 \degree$.

\subsection{Circuito RLC en paralelo}
\begin{figure}[H]
  \begin{center}
    \begin{circuitikz}
      \draw (0,0)
      to [sV, l=$v_i(t)$] (0,3)
      to [short, i^=$i_i(t)$] (3,3)
      to [short] (9,3)
      to [C, l_=$C$, v^=$v_C(t)$, i_=$i_C(t)$] (9, 0)
      to [short] (0,0);
      \draw (3,3)
      to[R, l_=$R$, v^=$v_R(t)$, i_=$i_R(t)$] (3,0);
      \draw (6,3)
      to[L, l_=$L$, v^=$v_L(t)$, i_=$i_L(t)$] (6,0);
    \end{circuitikz}
  \end{center}
\end{figure}

\begin{align*}
  v_i(t) &= V_0 \cos(\omega t)
\end{align*}

\begin{align*}
  i_R(t) &= \frac{V_0 \cos(\omega t)}{R} &
  i_L(t) &= \frac{V_0 \sin(\omega t)}{\omega L} &
  i_C(t) &= - \omega C V_0 \sin(\omega t)
\end{align*}

\begin{align*}
  i_i(t) &= i_R(t) + i_L(t) + i_C(t) \\
  &= \frac{V_0 \cos(\omega t)}{R} + \frac{V_0 \sin(\omega t)}{\omega L} - \omega C V_0 \sin(\omega t)\\
  &= \frac{V_0 \cos(\omega t)}{R} + \left[\frac{V_0}{\omega L} - \omega C V_0\right] \sin(\omega t)
\end{align*}

\begin{align*}
  P(t) &= \frac{V_0^2 \cos^{2}(\omega t)}{R} + V_0^2\left[\frac{1}{\omega L} - \omega C\right] \sin(\omega t) \cos(\omega t) \\
  &= \frac{V_0^2 + \cos(2 \omega t)}{2R} + \frac{V_0^2}{2} \left[\frac{1}{\omega L} - \omega C\right] \sin(2 \omega t)
\end{align*}
\begin{align*}
  \theta &= \alpha - \phi &
  \alpha &: \mathrm{fase\mspace{5mu} de\mspace{5mu}} i &
  \phi &: \mathrm{fase\mspace{5mu} de\mspace{5mu}} v
\end{align*}
\begin{align*}
  P_{Prom} &= \frac{V_0^2}{2} I_0 \cos(\theta)
\end{align*}

\subsection{Circuito RLC en serie}
\begin{figure}[H]
  \begin{center}
    \begin{circuitikz}
      \draw (0,0)
      to [sV, l=$v_i(t)$] (0,3)
      to[R, l_=$R$, v^=$v_R(t)$, i_=$i_R(t)$] (3,3)
      to[L, l_=$L$, v^=$v_L(t)$, i_=$i_L(t)$] (6,3)
      to [C, l_=$C$, v^=$v_C(t)$, i_=$i_C(t)$] (9,3)
      to [short, i^=$i_i(t)$] (9,0)
      to [short] (0,0);
    \end{circuitikz}
  \end{center}
\end{figure}


\begin{align*}
  v_i(t) &= V_0 \cos(\omega t)
\end{align*}

\begin{align*}
  i_R(t) &= \frac{V_0 \cos(\omega t)}{R} &
  i_L(t) &= \frac{V_0 \sin(\omega t)}{\omega L} &
  i_C(t) &= - \omega C V_0 \sin(\omega t)
\end{align*}

\begin{align*}
  i_i(t) &= i_R(t) + i_L(t) + i_C(t) \\
  &= \frac{V_0 \cos(\omega t)}{R} + \frac{V_0 \sin(\omega t)}{\omega L} - \omega C V_0 \sin(\omega t)\\
  &= \frac{V_0 \cos(\omega t)}{R} + \left[\frac{V_0}{\omega L} - \omega C V_0\right] \sin(\omega t)
\end{align*}

\begin{align*}
  P(t) &= \frac{V_0^2 \cos^{2}(\omega t)}{R} + V_0^2\left[\frac{1}{\omega L} - \omega C\right] \sin(\omega t) \cos(\omega t) \\
  &= \frac{V_0^2 + \cos(2 \omega t)}{2R} + \frac{V_0^2}{2} \left[\frac{1}{\omega L} - \omega C\right] \sin(2 \omega t)
\end{align*}
\begin{align*}
  \theta &= \alpha - \phi &
  \alpha &: \mathrm{fase\mspace{5mu} de\mspace{5mu}} i &
  \phi &: \mathrm{fase\mspace{5mu} de\mspace{5mu}} v
\end{align*}
\begin{align*}
  P_{Prom} &= \frac{V_0^2}{2} I_0 \cos(\theta)
\end{align*}
