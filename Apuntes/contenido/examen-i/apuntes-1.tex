%   Copyright 2016 Emilio Rojas
%
%   Licensed under the Apache License, Version 2.0 (the "License");
%   you may not use this file except in compliance with the License.
%   You may obtain a copy of the License at
%
%       http://www.apache.org/licenses/LICENSE-2.0
%
%   Unless required by applicable law or agreed to in writing, software
%   distributed under the License is distributed on an "AS IS" BASIS,
%   WITHOUT WARRANTIES OR CONDITIONS OF ANY KIND, either express or implied.
%   See the License for the specific language governing permissions and
%   limitations under the License.

\chapter{Temas Introductorios}

\section{Desfase, Adelanto, Atraso}
  \begin{figure}[H]
    \centering
    \includegraphics{contenido/tikz/desfase.tikz}
    \caption{Desfase de señales.}
    \label{img_desfase}
  \end{figure}
  En la figura \ref{img_desfase} las señales están desfasadas.
  \begin{itemize}
    \item $V_2$ adelanta a $V_1$ en $\phi$.
    \item $V_1$ se atrasa de $V_2$ en $\phi$.
  \end{itemize}
  Si $\phi=0\degree$, no hay desfase.

\section{Identidades Trigonométricas}
  A lo largo del documento se utilizan las siguientes identidades trigonométricas:
  \begin{multicols}{2}
    \begin{align}
      \sin(A \pm B) &= \sin(A) \cdot \cos(B) \pm \cos(A) \cdot \sin(B) \label{id_trig1}
      \\
      \cos(A \pm B) &= \cos(A) \cdot \cos(B) \mp \sin(A) \cdot \sin(B) \label{id_trig2}
    \end{align}

    \begin{align}
      \sin(A \pm 180\degree) &= -\sin(A) \label{id_trig3}
      \\
      \cos(A \pm 180\degree) &= -\cos(A) \label{id_trig4}
      \\
      \sin(A \pm 90\degree) &= \pm \sin(A) \label{id_trig5}
      \\
      \cos(A \pm 90\degree) &= \mp \cos(A) \label{id_trig6}
    \end{align}

    \begin{align}
      \cos^{2}(A) = \frac{1 + \cos(2 A)}{2} \label{id_trig7}
      \\
      \sin^{2}(A) = \frac{1 - \cos(2 A)}{2} \label{id_trig8}
    \end{align}

    \begin{align}
      \sin(2 A) = 2 \cos(A) \sin(A) \label{id_trig9}
    \end{align}
  \end{multicols}

\section{Elementos Pasivos}
  \begin{multicols}{2}
    \subsection*{Resistencia}
    \begin{figure}[H]
      \centering
      \begin{circuitikz}
        \draw (0,0)
        to [short, *-] (0.5, 0)
        to [R, l^=R, v_<=$v_R(t)$, i^=$i_R(t)$] (3.5, 0)
        to [short, -*] (4, 0);
      \end{circuitikz}
    \end{figure}

    \begin{align}
      v_R(t) &= R \cdot i_R(t) \label{v_r} \\
      i_R(t) &= \frac{v_R(t)}{R} \label{i_r}
    \end{align}
  \end{multicols}
  \noindent\rule{\textwidth}{0.4pt}

  \begin{multicols}{2}
    \subsection*{Inductor}
    \begin{figure}[H]
      \centering
      \begin{circuitikz}
        \draw (0,0)
        to [short, *-] (0.5, 0)
        to [L, l^=L, v_<=$v_L(t)$, i^=$i_L(t)$] (3.5, 0)
        to [short, -*] (4, 0);
      \end{circuitikz}
    \end{figure}

    \begin{align}
      v_L(t) &= L \cdot \frac{di_L(t)}{dt}\label{v_l} \\
      i_L(t) &= \frac{1}{L} \int v_L(t) dt \label{i_l}
    \end{align}
  \end{multicols}
  \noindent\rule{\textwidth}{0.4pt}

  \begin{multicols}{2}
    \subsection*{Capacitor}
    \begin{figure}[H]
      \centering
      \begin{circuitikz}
        \draw (0,0)
        to [short, *-] (0.5, 0)
        to [C, l^=C, v_<=$v_C(t)$, i^=$i_C(t)$] (3.5, 0)
        to [short, -*] (4, 0);
      \end{circuitikz}
    \end{figure}

    \begin{align}
      v_C(t) &= \frac{1}{C} \int i_C(t) dt \label{v_c} \\
      i_C(t) &= C \cdot \frac{dv_C(t)}{dt} \label{i_c}
    \end{align}
  \end{multicols}
  \noindent\rule{\textwidth}{0.4pt}

\section{Potencia Promedio}
  \begin{align}
    P(t) = i(t) \cdot v(t) = \frac{v^2(t)}{R} = i^2(t) \cdot R \label{pot} \\
    P_{Prom} = \frac{1}{T} \int_T P(t) dt \label{pot_prom}
  \end{align}

\subsection{Resistencia}
  \begin{wrapfigure}{l}{0.333\textwidth}
      \centering
      \begin{circuitikz}[yscale=2, xscale=1.2]
        \draw (0,0)
        to [sV, l=$v_i(t)$] (0,3)
        to [short, i^=$i_R(t)$] (2,3)
        to [R, l_=$R$, v^<=$v_R(t)$] (2, 0)
        to [short] (0,0);
      \end{circuitikz}
  \end{wrapfigure}

  Se define $v_i(t) = V_0 \cos(\omega t)$, se sabe $v_i(t) = v_R(t)$ y utilizando
  \ref{i_r}:

  \begin{equation*}
    i_R(t) = \frac{V_0 \cos(\omega t)}{R}
  \end{equation*}

  Utilizando \ref{pot}:
  \begin{equation*}
    P_R(t) = \frac{V_0^2 \cos^{2}(\omega t)}{R}
  \end{equation*}

  Utilizando \ref{id_trig7}:
  \begin{equation}
    P_R(t) = \frac{V_0^2 + V_0^2 \cos(2 \omega t)}{2 R} \label{pot_r}
  \end{equation}

  Ahora con \ref{pot_prom}:
  \begin{equation}
    P_{Prom_{R}} = \frac{V_0^2}{2 R} \label{pot_prom_r}
  \end{equation}

  \subsection{Inductor}
  \begin{wrapfigure}{l}{0.333\textwidth}
      \centering
      \begin{circuitikz}[yscale=2.1, xscale=1.2]
        \draw (0,0)
        to [sV, l=$v_i(t)$] (0,3)
        to [short, i^=$i_L(t)$] (2,3)
        to [L, l_=$L$, v^<=$v_L(t)$] (2, 0)
        to [short] (0,0);
      \end{circuitikz}
  \end{wrapfigure}

  Se define $v_i(t) = V_0 \cos(\omega t)$, se sabe $v_i(t) = v_L(t)$ y utilizando
  \ref{i_l}:

  \begin{equation*}
    i_L(t) = \frac{V_0 \sin(\omega t)}{\omega L} = \frac{V_0 \cos(\omega t - 90 \degree)}{\omega L}
  \end{equation*}

  Utilizando \ref{pot}:
  \begin{equation*}
    P_L(t) = \frac{V_0^2 \sin(\omega t) \cos(\omega t)}{\omega L}
  \end{equation*}

  Utilizando \ref{id_trig9}:
  \begin{equation}
    P_L(t) = \frac{V_0^2 \sin(2 \omega t)}{2 \omega L} \label{pot_l}
  \end{equation}

  Ahora con \ref{pot_prom}:
  \begin{equation}
    P_{Prom_{L}} = 0 \label{pot_prom_l}
  \end{equation}

  \textbf{Nota:} El inductor adelanta la corriente $90 \degree$.

  \newpage
  \subsection{Capacitor}
  \begin{wrapfigure}{l}{0.333\textwidth}
      \centering
      \begin{circuitikz}[yscale=2.1, xscale=1.2]
          \draw (0,0)
          to [sV, l=$v_i(t)$] (0,3)
          to [short, i^=$i_C(t)$] (2,3)
          to [C, l_=$C$, v^<=$v_C(t)$] (2, 0)
          to [short] (0,0);
        \end{circuitikz}
  \end{wrapfigure}

  Se define $v_i(t) = V_0 \cos(\omega t)$, se sabe $v_i(t) = v_C(t)$ y utilizando
  \ref{i_c}:

  \begin{equation*}
    i_C(t) = - \omega C V_0 \sin(\omega t) = \omega C V_0 \cos(\omega t + 90 \degree)
  \end{equation*}

  Utilizando \ref{pot}:
  \begin{equation*}
    P_C(t) = - \omega C V_0^2 \sin(\omega t) \cos(\omega t)
  \end{equation*}

  Utilizando \ref{id_trig9}:
  \begin{equation}
  P_C(t) = - \frac{\omega C V_0^2 \sin(2 \omega t)}{2} \label{pot_c}
  \end{equation}

  Ahora con \ref{pot_prom}:
  \begin{equation}
    P_{Prom_{C}} = 0 \label{pot_prom_c}
  \end{equation}

  \textbf{Nota:} El capacitor atrasa la corriente $90 \degree$.

  \subsection{Circuito RLC en paralelo}
  \begin{figure}[H]
    \begin{center}
      \begin{circuitikz}
        \draw (0,0)
        to [sV, l=$v_i(t)$] (0,3)
        to [short, i^=$i_i(t)$] (3,3)
        to [short] (9,3)
        to [C, l_=$C$, v^<=$v_C(t)$, i_=$i_C(t)$] (9, 0)
        to [short] (0,0);
        \draw (3,3)
        to[R, l_=$R$, v^<=$v_R(t)$, i_=$i_R(t)$] (3,0);
        \draw (6,3)
        to[L, l_=$L$, v^<=$v_L(t)$, i_=$i_L(t)$] (6,0);
      \end{circuitikz}
    \end{center}
  \end{figure}

  \begin{equation*}
    v_i(t) = V_0 \cos(\omega t)
  \end{equation*}

  \begin{align*}
    i_R(t) &= \frac{V_0 \cos(\omega t)}{R} &
    i_L(t) &= \frac{V_0 \sin(\omega t)}{\omega L} &
    i_C(t) &= - \omega C V_0 \sin(\omega t)
  \end{align*}

  \begin{align*}
    i_i(t) &= i_R(t) + i_L(t) + i_C(t) \\
    &= \frac{V_0 \cos(\omega t)}{R} + \frac{V_0 \sin(\omega t)}{\omega L} - \omega C V_0 \sin(\omega t)\\
    &= \frac{V_0 \cos(\omega t)}{R} + V_0\left[\frac{1}{\omega L} - \omega C \right] \sin(\omega t)
  \end{align*}

  \begin{align*}
    P(t) &= \frac{V_0^2 \cos^{2}(\omega t)}{R} + V_0^2\left[\frac{1}{\omega L} - \omega C\right] \sin(\omega t) \cos(\omega t) \\
    &= \frac{V_0^2 + V_0^2 \cos(2 \omega t)}{2R} + \frac{V_0^2}{2} \left[\frac{1}{\omega L} - \omega C\right] \sin(2 \omega t)\\
    &= \frac{V_0^2}{2R} + \frac{V_0^2 \cos(2 \omega t)}{2R} + \frac{V_0^2}{2} \left[\frac{1}{\omega L} - \omega C\right] \sin(2 \omega t)
  \end{align*}
  \begin{align*}
    \theta &= \alpha - \phi &
    \alpha &: \mathrm{fase\mspace{5mu} de\mspace{5mu}} i &
    \phi &: \mathrm{fase\mspace{5mu} de\mspace{5mu}} v
  \end{align*}
  \begin{align*}
    P_{Prom} &= \frac{V_0^2}{2} I_0 \cos(\theta)
  \end{align*}

  \subsubsection*{$i(t)$ del circuito}

  \begin{equation*}
    i_i(t) = \sqrt{\frac{V_0^2}{R^2} + \left(\frac{V_0}{\omega L} - \omega CV_0\right)^2} \cos(\omega t - \beta)
  \end{equation*}

  \begin{equation*}
    A \cos(\omega t) + B \sin(\omega t) = C \cos(\omega t - \theta)
  \end{equation*}

  \begin{equation*}
    C = \sqrt{A^2 + B^2}, \theta = \tan^{-1}\left(\frac{A}{B} \right)
  \end{equation*}

  \begin{equation*}
    \beta = tan^{-1}\left(\frac{\frac{V_0}{R}}{\frac{V_0}{\omega L} - \omega C V_0}\right)
  \end{equation*}

  \begin{center}
    \begin{tikzpicture}
      \draw
      (0,0) coordinate (a)
      -- (1.5,0) coordinate (label_adjacent)
      node[below] {$\frac{V_0}{\omega L} - \omega C V_0$}
      -- (3,0) coordinate (b)
      -- (1.5, 1.5) coordinate (label_hypotenuse)
      node[above right] {$\sqrt{\frac{V_0^2}{R^2} + \left(\frac{V_0}{\omega L} - \omega CV_0\right)^2}$}
      -- (0,3) coordinate (c)
      -- (0,1.5) coordinate (label_opposite)
      node[left] {$\frac{V_0}{R}$}
      -- (0,0) coordinate (d) node[above right] {}
      pic["$\beta$", draw=black, angle eccentricity=1.2, angle radius=1cm]
      {angle=c--b--a};
    \end{tikzpicture}
  \end{center}

  Aplicando fórmula:
  \begin{equation*}
    v_i(t) = V_0\cos(\omega t + 0 \degree)
  \end{equation*}
  \begin{equation*}
    \phi = \beta - 90 \degree \to \cos(\theta) = \cos(\beta - 90 \degree) = \sin(\beta)
  \end{equation*}

  \begin{equation*}
    \sin(\beta) = \frac{\frac{V_0}{R}}{\sqrt{\frac{V_0^2}{R^2} + \left(\frac{v_0}{\omega L} - \omega C V_0 \right)^2}}
  \end{equation*}

  \begin{equation}
    P_{Prom_{Paralelo}} = \frac{V_0 \sqrt{\frac{V_0^2}{R^2} + \left(\frac{V_0}{\omega L} - \omega C V_0 \right)^2}}{2 \sqrt{\frac{V_0^2}{R^2} + \left(\frac{V_0}{\omega L} - \omega C V_0 \right)^2}} \cdot \frac{V_0}{R} = \frac{V_0^2}{2R} \label{pot_prom_paral}
  \end{equation}
